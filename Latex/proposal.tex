\documentclass[pra, reprint, A4 paper, bibnotes]{revtex4-1}  
% Replace reprint with preprint if you want a one column layout
% Add superscriptaddress if there's more than one address (i.e., a collaboration with others)
% For superscript references, replace pra with prb and add class option citeautoscript
% Include 11pt after reprint if you want to define font size

% Font and input encoding
\usepackage[T1]{fontenc}
\usepackage[utf8]{inputenc} % Enables direct input of Unicode characters
\usepackage[australian]{babel}
\usepackage{lipsum}

% This seems to be necessary to get sensible margins on A4 paper. Alternatively, switch to letterpaper in the class options:
\usepackage[a4paper,centering,hmargin=1.75cm,vmargin=1.8cm]{geometry} 

% Packages
\usepackage{amsmath,amssymb,graphicx,bm,microtype} % Stuff you need in every paper
\usepackage[dvipsnames]{xcolor} % Colour
\usepackage{booktabs} % Nice tables
\usepackage{siunitx} \sisetup{exponent-product=\cdot} % SI units, with \cdot instead of \times
\usepackage{braket} % For \bra, \ket, \braket, \Bra, \Ket, \Braket
\usepackage[version=4]{mhchem} % Chemical formulas
\usepackage[italic,thinc]{esdiff} % Derivatives
\usepackage[colorlinks,allcolors=blue!50!black]{hyperref} % Hyperlinks. The options set up coloured links within the paper, which is important if you're submitting to arXiv to avoid the ugly neon-coloured boxes they impose by default. If you want to make the links black, use [hidelinks] instead
\usepackage[all]{hypcap} % Makes sure that hyperref links to floats jump to the top of the float and not to the top of the caption
\usepackage{cleveref} % Clever referencing
\usepackage{flafter} % Avoids tables floating before figures
\usepackage[section]{placeins} %Avoids figures floating to different sections
\usepackage{textcomp}
\usepackage{floatrow}
% Table float box with bottom caption, box width adjusted to content
\newfloatcommand{capbtabbox}{table}[][\FBwidth]
\usepackage{appendix}
\def\bibsection{\section*{\refname}} %removes line, adds references heading

% Layout
\renewcommand{\thetable}{\arabic{table}} % Arabic numbering for tables
\renewcommand{\tabcolsep}{4pt} % A bit more spacing between table columns

% Absolute values, averages, commutators
\newcommand{\abs}[1]{\lvert #1 \rvert}
\newcommand{\Abs}[1]{\left\lvert #1 \right\lvert}
\newcommand{\norm}[1]{\lVert #1 \rVert}
\newcommand{\Norm}[1]{\left\lVert #1 \right\rVert}
\newcommand{\avg}[1]{\langle #1 \rangle}
\newcommand{\Avg}[1]{\left\langle #1 \right\rangle}
\newcommand{\comm}[2]{[#1,#2]}
\newcommand{\Comm}[2]{\left[#1,#2\right]}
\newcommand{\anticomm}[2]{\{#1,#2\}}
\newcommand{\Anticomm}[2]{\left\{#1,#2\right\}}

% Vectors
\DeclareMathSymbol{\ii}{\mathalpha}{letters}{"10} % upright dotless i
\DeclareMathSymbol{\jj}{\mathalpha}{letters}{"11} % upright dotless j
\renewcommand{\vec}[1]{\bm{\mathrm{#1}}}
\newcommand{\unitvec}[1]{\vec{\hat{#1}}}
\newcommand{\del}{\nabla}
\newcommand{\grad}{\nabla}
\renewcommand{\div}{\nabla \cdot}
\newcommand{\curl}{\nabla \times}

% Common operators
\DeclareMathOperator{\tr}{tr}
\DeclareMathOperator{\Tr}{Tr}
\let\Re\relax \DeclareMathOperator{\Re}{Re}
\let\Im\relax \DeclareMathOperator{\Im}{Im}

% Miscellaneous maths
\renewcommand{\d}{\mathop{}\!d} % d for integrals with automatic correct spacing
\newcommand{\half}{\frac{1}{2}}
\newcommand{\dg}{\dagger}
\newcommand{\eff}{\mathrm{eff}}
\newcommand{\hc}{\mathrm{h.c.}}

% Horrid kludge for upright Greek letters
\newsavebox{\foobox}
\newcommand{\slantbox}[2][0]{\mbox{%
        \sbox{\foobox}{#2}%
        \hskip\wd\foobox
        \pdfsave
        \pdfsetmatrix{1 0 #1 1}%
        \llap{\usebox{\foobox}}%
        \pdfrestore
}}
\newcommand\unslant[2][-.25]{\slantbox[#1]{$#2$}}

% Comments
\newcommand{\cik}[1]{\textit{\textcolor{Red}{KG: #1}}}
 % Define your own and pick your own colour!
% To hide all comments uncomment the following lines. However, all comments should be removed manually from submitted versions.
%\renewcommand{\cik}[1]{}

%\usepackage{gensymb}
\usepackage{booktabs}
\usepackage{subfig}
\usepackage{tabularx}
\graphicspath{{./Pictures/}}

\usepackage{fancyhdr} % Custom headers and footers
%\fancyhead[R]{} % No page header - if you want one, create it in the same way as the footers below
%\fancyhead[CO,CE]{---Draft---}
%\fancyfoot[L]{} % Empty left footer
%\fancyfoot[C]{\thepage} % Page numbering for center footer
%\fancyfoot[R]{} % Empty right footer
%\renewcommand{\headrulewidth}{0pt} % Remove header underlines
%\renewcommand{\footrulewidth}{0pt} % Remove footer underlines
%\setlength{\headheight}{13.6pt} % Customize the height of the header
\setlength\parindent{0pt} % Removes all indentation from paragraphs - comment this line for an assignment with lots of text
\fancyhead{} % clear all header fields
\fancyhead[L]{The University of Queensland\\PHYS3900 - Perspectives in Physics Research}
\fancyfoot{} % clear all footer fields
\pagestyle{fancy}


%\captionsetup{belowskip=12pt,aboveskip=4pt}
%\captionsetup[table]{justification=justified,singlelinecheck=false}
\newcommand\T{\rule{0pt}{2.6ex}} % Top strut
\newcommand\B{\rule[-1.2ex]{0pt}{0pt}} % Bottom strut
%\nocite{*}
%----------------------------------------------------------------------------------------
%	TITLE SECTION
%----------------------------------------------------------------------------------------
\begin{document}
\title{Project Proposal\\A Fly-Through Visualisation of the Dark Energy Survey}
\author{\textbf{Katherine Gray - 43911064}\\ Supervisor: Tamara Davis}

\maketitle % Print the title

\thispagestyle{fancy} %Bring back fancyhdr
%%%%%%%%%%%%%%%%%%%%%%%%%%%%%%%%%%%%%%%%%%%%%%%%%%%%%%%%%%%%%%%%%
\section{Introduction}
\textbf{Background}\\
The Dark Energy Survey (DES) is an astronomical imaging survey of the Southern Sky that has been running as an international collaboration since 2013. The survey has been collecting and categorising data from thousands of supernovae and hundreds of millions of galaxies, with the ultimate goal of investigating dark energy and the acceleration of the universe ~\cite{dark_energy_survey_collaboration_dark_2016}.The survey is due to finish this year and my proposed project is to visualise the data collected by creating a video emulating an observer's point of view flying from Earth to the furthest point of the survey. \\

\textbf{Aim}\\
The primary aim of this project is to create an accurate and engaging video which can be used to promote the Dark Energy Survey in the media.\\

\textbf{Previous Work}\\
Some DES visualisations have already been created~\cite{samuel_hinton_+ozdes+sdss_2017, ralf_kaehler_year_2017}, however the fly-through visualisation technique and the fact that I will have access to a complete set of data will distinguish my project from previous work. I am personally looking forward to working in astrophysics because I have not yet had an opportunity to study related courses as part of my degree but am intrigued by the subject. I am also interested in this work because I am heading into a career in data analytics next year and the ability to visualise data well is a valuable skill in the industry and one that I wish to develop. 
 %%%%%%%%%%%%%%%%%%%%%%%%%%%%%%%%%%%%%%%%%%%%%%%%%%%%%%%%%%%%%%%%%
\section{Project Significance}
It is estimated that 44 quintillion ($10^{18}$) bytes of data was produced every single day in 2016~\cite{micro_focus_growth_2017}. At its core, the Dark Energy Survey is just another source of data producing, for comparison, up to 2.5 trillion ($10^{12}$) bytes of data a night~\cite{the_dark_energy_survey_survey_2018}. Creating a visualisation of this data is important because it provides a means by which people can absorb, understand and use the survey. Videos, in particular, are effective tools for communicating because compared to text they are easier to absorb, more attention-grabbing, and more memorable~\cite{chan_video_2010}. This project will produce a video that is visually pleasing, scientifically accurate and can be appreciated by individuals who aren't scientists, making it a perfect candidate for use in a promotional campaign.\\

Already, the DES organisation has made multiple appearances in the media having been being picked up by various news outlets worldwide and staying active on Youtube, Twitter and Facebook. They have shared stories that range from DES scientist profiles, to progress updates and short educational videos - again, content that is deliberately easy to digest, but exciting. These publicity efforts, augmented by my contribution, can continue to help society understand the current status of scientific research and its important and relevance to the entire community. In turn, this encourages younger generations to become involved in science, ensures continued support and funding for worthy research and cultivates an educated population capable of making well-informed decisions ~\cite{jucan_power_2014, julie_gould_importance_2014}. 
%%%%%%%%%%%%%%%%%%%%%%%%%%%%%%%%%%%%%%%%%%%%%%%%%%%%%%%%%%%%%%%%%
\section{Method}
\textbf{Process and Technologies}\\
I will not have immediate access to the Dark Energy Survey data because this will have to be properly authenticated. In the mean time, I plan on using an initial set of galaxy data that has already been released to the public. The raw data of each galaxy location is given in equatorial coordinates and will be converted to Cartesian and spherical coordinates to facilitate easy manipulation. Using Python 3.6 and the numpy and matplotlib libraries, I will implement the ability to translate the data points to reflect a moving observer's point of view. For each point, this will include not only updating the point's coordinates, but also changing its size and opacity to imitate the effects of perspective. I will then produce multiple pngs of the data represented as if the observer is following a continuous flight path and stitch this sequence of images together to create a video using FFmpeg software.\\

I will be using git as my version control system and GitHub to host my repository online. These technologies will back up my work for me remotely, allow me to keep a complete history of changes and make sharing my work with other potential contributors easy. I have used git and GitHub for many of projects before so will not need to upskill in this area, and, like the other technologies I will be using in my project, are completely free to use.\\

\textbf{Timeline}

\begin{table}[H]
\begin{tabular}{ll}
\hline
Week & Description                                                                                             \\\hline
1    & Finalise project, set up working environment\\
2    & Access data set, implement translation functionality                                      \\
3    & Write project proposal, test and refine \\&translation functionality                                  \\
4    & Create output sequence of pngs, experiment  \\&with plotting sizes and colours                              \\
5    & Receive access to full dataset, clean it and load it in                                                 \\
6    & Upskill in FFmpeg, create a video of the fly-though                                                     \\
7    & Experiment with fly-though paths, start drafting \\&report                                                 \\
8    & Refine research report, generate final version      \\\hline                                                   
\end{tabular}
\end{table}
            
%%%%%%%%%%%%%%%%%%%%%%%%%%%%%%%%%%%%%%%%%%%%%%%%%%%%%%%%%%%%%%%%%
\section{Expected Outcomes}
I expect that the video I produce will be used by the Dark Energy Survey organisation in multiple ways. It is very likely that it will be uploaded to their YouTube channel, and shared on their other social media platforms. Similar videos released in this way have recorded up to two thousand views and sparked meaningful conversations in the comments. It is also likely that the video I produce will be used as stock footage for any televised or online segments reporting on DES and its progress. This will allow viewers of these reports to get a feel for the scale of the project and see some real results before in-depth analysis has even begun.\\

The fly-through footage should therefore generate quite a large audience, ideally capturing the interest of as many people as possible and effecting the positive outcomes of good science communication. 
%%%%%%%%%%%%%%%%%%%%%%%%%%%%%%%%%%%%%%%%%%%%%%%%%%%%%%%%%%%%%%%%%
\bibliographystyle{apsrev4-1-fixed}
\bibliography{proposal}
\end{document}